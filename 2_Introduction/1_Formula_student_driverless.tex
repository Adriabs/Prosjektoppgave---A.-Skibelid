\section{Formula student driverless}

This project thesis is done as part of Revolve NTNU, a team competing in various competitions around Europe in what is known as Formula Student. Formula Student is often referred to as the worlds largest engineering competition, while in reality it consists of several competitions spread around the globe, one of the msot influential of which is Formula Student Germany (FSG). FSG 2018 had over 300 students in 66 teams from 8 different countries. 

Each competition follows more or less the same recipe; build a car that gets through, the very strict, scrutineering, where every aspect of the car must follow an intense safety and quality protocol demanded by the competition. If the car passes it competes in one of three classes: combustion, electric or driverless. 

This implies both static and dynamic events, where the static are focused on cost, documentation of the engineering process, and so on. The dynamic events for the combustion and electric class encompasses showing how energy efficient the car is, how well it accelerates, corners and so on. 

The driverless class is a rather new one, only being introduced at FSG in 2017. Revolve NTNU had it's first driverless team in 2018. Each driverless car is tested in four dynamic events: Acceleration, Skidpad, autocross and trackdrive. Each of the events are to be done without a driver in the car and without signals being sent to the car from the team in any way. In all events the goal is to drive as fast as possible without knocking any delimiters, which are blue cones on the left side and yellow cones on the right. Knocking a delimiter gives a time penalty.

The acceleration event is just a 75 meter straight path, very much akin to a drag race. Skidpad has the car driving in a figure eight pattern, taking two loops on each circle, only being timed on the second run of each circle. Autocross is one round of an unknown track, where it isn't allowed to do any mapping before starting the event. Trackdrive is the same as autocross, only now driving 10 rounds. 

Each event therefore requires somewhat different approaches. What is however constant is the need for good state estimation, detection, localization and mapping, so the car can drive fast without hitting any delimiters.