\section{SLAM}

The SLAM backend was implemented using the GTSAM factor graph framework\cite{GTSAM}, and more specifically using its iSAM2\cite{iSAM2} library. As explained above, iSAM2 avoids periodic batch operations, thus keeping the run time down. This property is crucial for a system like an autonomous race car, as the high speed of the car means an accurate map is useless if it arrives too late. 

The optimisation still uses some time, much more than the frontend. Therefore, in order for the frontend to be able to work without having to wait for the optimisation to finish, the frontend and backend are separated into two threads. This makes it possible to both optimise the graph and handle new incoming detections at the same time. 

The disadvantage is however that the interplay between the two threads need to be controlled. This is done by having the two threads communicate through variables that are locked by mutexes, i.e. only allowing one thread to operate on the variables at a time. The map and odometry information, as well as the list of new measurements and new landmarks being sent from the frontend to the backend are locked by mutexes. This makes sure that no information is lost or corrupted by simultaneous operations on the same variable. 

Another concern that had to be addressed was how the system could use both the estimated pose of the vehicle in the odometry frame, and the correction from the SLAM backend on the transformation between the odometry frame and the map frame. This was done using the \gls{ROS} tf framework. It gives access to a lookupTransform function that gives the transformation between any two frames that have all the needed transformations posted on the tf server. Thus the newest estimate of the body frame could be found by querying for the transform between the map and body frames. 

\iffalse
\subsubsection{BearingRange factor}

The landmark measurements added to the factor graph are added as bearingrange factors, even though they are really measurements of an $x$ and a $y$ position in the body frame. This is because otherwise the factor had to be implemented ourselves, whereas the BearingRange factor comes with GTSAM. How this affects the system is a bit unknown, but it does seem less than ideal, and is probably something that should be fixed, if only there was time.
\todo[inline]{Remove this section?}
\fi
%\subsection{}