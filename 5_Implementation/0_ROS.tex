This part tries to shed light on some practical details around the pipeline and developed algorithms. 

\section{ROS}

For the whole autonomous pipeline, the system uses \gls{ROS}\cite{ROS}. It claims to be a "meta-operating system", providing a wealth of services that a modern robot system might need, like message-passing between processes, facilitating debugging, visualisation, low level device control and packet management. It also includes a standardised framework for working with coordinate system transformations and an abundance of standard and user-made packages for all sorts of operations one might need. 

What the team used it for most is package management, providing a unified system for building the whole pipeline, the standardised message passing system and the transformation framework. It also provides an easy interface between the software and many of the car's sensors, like the cameras, converting sensor data into standard message types that are easy to use by the rest of the system. Finally it gives a nice way of visualising data, using \gls{ROS}'s RVIZ, and for debugging the communication between the different nodes in the pipeline, using rqt and especially rqt\_graph.